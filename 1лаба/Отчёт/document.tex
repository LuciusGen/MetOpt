\documentclass[a4]{article}
\pagestyle{myheadings}

%%%%%%%%%%%%%%%%%%%
% Packages/Macros %
%%%%%%%%%%%%%%%%%%%
\usepackage{mathrsfs}


\usepackage{fancyhdr}
\pagestyle{fancy}
\lhead{}
\chead{}
\rhead{}
\lfoot{}
\cfoot{} 
\rfoot{\normalsize\thepage}
\renewcommand{\headrulewidth}{0pt}
\renewcommand{\footrulewidth}{0pt}
\newcommand{\RomanNumeralCaps}[1]
{\MakeUppercase{\romannumeral #1}}

\usepackage{amssymb,latexsym}  % Standard packages
\usepackage[utf8]{inputenc}
\usepackage[russian]{babel}
\usepackage{MnSymbol}
\usepackage{amsmath,amsthm}
\usepackage{indentfirst}
\usepackage{graphicx}%,vmargin}
\usepackage{graphicx}
\graphicspath{{pictures/}} 
\usepackage{verbatim}
\usepackage{color}









\DeclareGraphicsExtensions{.pdf,.png,.jpg}% -- настройка картинок

\usepackage{epigraph} %%% to make inspirational quotes.
\usepackage[all]{xy} %for XyPic'a
\usepackage{color} 
\usepackage{amscd} %для коммутативных диграмм


\newtheorem{Lemma}{Лемма}[section]
\newtheorem{Proposition}{Предложение}[section]
\newtheorem{Theorem}{Теорема}[section]
\newtheorem{Corollary}{Следствие}[section]
\newtheorem{Remark}{Замечание}[section]
\newtheorem{Definition}{Определение}[section]
\newtheorem{Designations}{Обозначение}[section]




%%%%%%%%%%%%%%%%%%%%%%%% 
%Сношение с оглавлением% 
%%%%%%%%%%%%%%%%%%%%%%%% 
\usepackage{tocloft} 
\renewcommand{\cftdotsep}{2} %частота точек
\renewcommand\cftsecleader{\cftdotfill{\cftdotsep}}
\renewcommand{\cfttoctitlefont}{\hspace{0.38\textwidth} \LARGE\bfseries} 
\renewcommand{\cftsecaftersnum}{.}
\renewcommand{\cftsubsecaftersnum}{.}
\renewcommand{\cftbeforetoctitleskip}{-1em} 
\renewcommand{\cftaftertoctitle}{\mbox{}\hfill \\ \mbox{}\hfill{\footnotesize Стр.}\vspace{-0.5em}} 
\renewcommand{\cftsubsecfont}{\hspace{1pt}} 
\renewcommand{\cftparskip}{3mm} %определяет величину отступа в оглавлении
\setcounter{tocdepth}{5} 




\addtolength{\textwidth}{0.7in}
\textheight=630pt
\addtolength{\evensidemargin}{-0.4in}
\addtolength{\oddsidemargin}{-0.4in}
\addtolength{\topmargin}{-0.4in}

\newcommand{\empline}{\mbox{}\newline} 
\newcommand{\likechapterheading}[1]{ 
	\begin{center} 
		\textbf{\MakeUppercase{#1}} 
	\end{center} 
	\empline} 

\makeatletter 
\renewcommand{\@dotsep}{2} 
\newcommand{\l@likechapter}[2]{{\bfseries\@dottedtocline{0}{0pt}{0pt}{#1}{#2}}} 
\makeatother 
\newcommand{\likechapter}[1]{ 
	\likechapterheading{#1} 
	\addcontentsline{toc}{likechapter}{\MakeUppercase{#1}}} 





\usepackage{xcolor}
\usepackage{hyperref}
\definecolor{linkcolor}{HTML}{000000} % цвет ссылок
\definecolor{urlcolor}{HTML}{AA1622} % цвет гиперссылок

\hypersetup{pdfstartview=FitH,  linkcolor=linkcolor,urlcolor=urlcolor, colorlinks=true}



\def \newstr {\medskip \par \noindent} 



\begin{document}
	\def\contentsname{\LARGE{Содержание}}
	\thispagestyle{empty}
	\begin{center} 
		\vspace{2cm} 
		{\Large \sc Санкт-Петербургский Политехнический Университет}\\
		\vspace{2mm}
		{\Large\sc Петра Великого}\\
		\vspace{1cm}
		{\large \sc Институт прикладной математики и механики\\ 
			\vspace{0.5mm}
			\textsc{}}\\ 
		\vspace{0.5mm}
		{\large\sc Кафедра $"$Прикладная математика$"$}\\
		\vspace{15mm}
		
		
		{\sc \textbf{
			Отчёт по лабораторной работе №$1$\\
			"Решение задачи линейного программирования симплекс-методом"}
			\vspace{6mm}
			
		}
		\vspace*{2mm}
		
		
		\begin{flushleft}
			\vspace{4cm}
			\sc Выполнили студенты:\\
			\sc Салихов С.\\
			\sc Шарапов С.\\
			\sc Мальцов Д.\\
			\sc группа: 3630102/70401\\
			\vspace{1cm}
			\sc Проверил:\\
			\sc к.ф-м.н.\\
			\sc Родионова Е.А.
			\vspace{20mm}
		\end{flushleft}
	\end{center} 
	\begin{center}
		\vfill {\large\textsc{Санкт-Петербург}}\\ 
		2020 г.
	\end{center}
	
	\newpage
	\tableofcontents
	\newpage
	%\begin{center}
	%\begin{abstract} 
	
	%\end{abstract}
	
	%\end{center}
	
	\section{Постановка задачи}
		Требуется с помощью симплекс-метода с использованием начального приближения методом искусственного базиса и метода крайних точек решить прямую и двойственную задачу линейного программирования, включающую 4 переменных, 1 имеет ограничение на знак, и состоящую из 2 неравенств разных знаков и 2 равенств.\\
		Также требуется написать алгоритм позволяющий восстановить решение прямой задачи по решению двойственной.\\
		\subsection{Постановка задачи ЛП}
		Составим задачу подходящую под наши требования:\\
		\begin{equation*}
			\begin{cases}
				x_1 + x_3 = 5\\
				x_2 + x_3 = 4\\
				4x_1 + 2x_4 >= 7\\
				2x_1 + 4x_2 <= 8\\
				x_1>=0
			\end{cases}
		\end{equation*}
		
		\begin{center}
			$C = x_1 + 2x_2 + 3x_3 + 4x_4$ -> min
		\end{center}
		
	\section{Условие применимости}
		Любая задача ЛП сводится к канонической форме. Каноническая форма имеет следующий вид:\\
		AX = b\\
		X >= 0, b>=0
		C = <$c_i$, x> -> extr
		Сведём нашу систему к каноническому виду, для применения методов:\\
		
		\begin{equation*}
			\begin{cases}
				x_1 + x_4 - x_5 = 5\\
				x_2 - x_3 + x_4 - x_5 = 4\\
				4x_1 + 2x_6 - 2x_7 - x_8 = 7\\
				2x_1 + 4x_2 - 4x_3 + x_9 = 8\\
				x_1, .., x_9 >=0
			\end{cases}
		\end{equation*}
		
		\begin{center}
			$C = x_1 + 2x_2 - 2x_3 + 3x_4  - 3x_5 + 4x_6 - 4x_7$ -> min
		\end{center}
	\section{Описание алгоритма}
		\subsection{Метод искуственного базиса}
			Метод заключается в специальном выборе начального приближения(Реализуется, если матрица полного ранга). Алгоритм выбора данного приближения следующий:\\
			1)Вводится вспомогательная задача: $\sum_{i \in M}$y[i] -> min\\
			A[M,N]X[N] + E[M,M]y[M] = b[M]\\
			x[N]>=0\\
			y[N]>=0\\
			2)Начальный опорный вектор : x[N] = 0\\
			y[N] = b[M], где b[M] >= 0\\
			В качестве базиса столбца E[M,M]\\
			3)По завершении работы симплекс-метода отбрасываем y[N], которые мы ввели и получаем решение исходной задачи.
		\subsection{Симплекс-метод}
			Симплекс-метод позволяет переходить от одного опорного вектора к другому так, что значение целевой функции не увеличивается.\\
			В результате не более, чем за $C^m_n$(максимальное количество опорных векторов) шагов достигается оптимальное значение, либо опровергается его существование.
			
		\subsection{Метод крайних точек}
			В свою очередь метод крайних точек заключается в переборе возможных базисных векторов и выявление того, при котором функция цели будет наименьшей. Своего рода это полный переборный алгоритм.
		
		\subsection{Восстановление решения по решению двойственной задачи}
			Обозначим через $C_\sigma$=($c_{i1}, .. , c_{im}$) вектор строку, составленную из коэффициентов при неизвестных в целевой функции задачи. А через $P^-1$ - матрицу обратную матрице Р, где Р состоит из компонент векторов $P_{i1}$,.., $P_{im}$\\
			Тогда имеет место теорема: Если основная задача имеет оптимальный план $X^*$, то $Y^*$ = $C_\sigma P^-1$ является оптимальным планом двойственной задачи.
	\section{Результаты}
		\subsection{Результат решения симплекс методом с начальным приближением методом искуственного базиса}
			\subsubsection{Начальный базис вектор полученный методом искуственного базиса}
				В нашей задаче базис будет выглядеть следующим образом:\\
				X0 = [0, 0, 0, 0, 0, 0, 0, 0, 0, 5, 4, 7, 8]
			\subsubsection{Интерпретация решения}
				Условие оптимальности полученного решения:\\
				· Если задача на максимум – в строке функционала нет отрицательных коэффициентов (т.е. при любом изменении переменных значение итогового функционала расти не будет).\\
				· Если задача на минимум – в строке функционала нет положительных коэффициентов (т.е. при любом изменении переменных значение итогового функционала уменьшаться не будет).
			
			\subsection{Ответ в поставленной задаче}
				X = [2.0, 1.0, 0.0, 3.0, 0.0, 0.0, 0.5, 0.0, 0.0]\\
				
				N = [0, 1, 3, 6]\\
				
				Оптимальное значение функции цели: C = 11.0
			\subsection{Результат методом крайних точек}
				X = [1.9999999999999998, 1.0000000000000004, 0.0, 3.0, 0.0, 0.0, 0.49999999999999956, 0.0, 0.0]\\
				
				Оптимальное значение функции цели: C = 11.000000000000002
			\subsection{Двойственная задача}
			Двойственная к поставленной задаче(из канонического вида поставленной задачи) в каононическом виде:
			\begin{equation*}
				\begin{cases}
					x_0 -x_1 + 4x_4 - 4x_5 + 2x_6 - 2x_7 + x_8 = 1\\
					x_2 - x_3 + 4x_6 - 4x_7 + x_9 = 2\\
					x_2 -x_3 + 4x_6 - 4x_7 - x_{10} = 2\\
					x_0 -1x_1 + x_2 + -x_3 + x_{11} = 3\\
					x_0 - x_1 + x_2 -x_3 - x_{12} = 3\\
					2x_4 - 2x_5 + x_{13} = 4\\
					2x_4 -2x_5 - x_{14} = 4\\
					-x_4 + x_5 + x_{15} = 0\\
					x_6 -x_7 + x_{16} = 0\\
					x_0.. x_{16} >= 0 
				\end{cases}
			\end{equation*}
			$С = 5*x_0 - 5*x_1 + 4*x_2 - 4*x_3 + 7*x_4 - 7*x_5 + 8*x_6 - 8*x_7->max$
			
			\subsection{Результаты двойственной задачи}
				\subsubsection{Симплекс методом}
					X = [0.0, 4.333333333333334, 7.333333333333333, 0.0, 2.0, 0.0, 0.0, 1.3333333333333333, 0.0, 0.0, 0.0, 0.0, 0.0, 0.0, 0.0, 2.0, 1.3333333333333335]\\
					
					N = [1, 2, 4, 7, 15, 16]\\
					
					Оптимальное значение функции цели: C = 10.99999999999999
				\subsubsection{методом крайних точек}
					X = [0.0, 4.333333333333334, 7.333333333333333, 0.0, 2.0, 0.0, 0.0, 1.3333333333333333, 0.0, 0.0, 0.0, 0.0, 0.0, 0.0, 0.0, 2.0, 1.3333333333333335]\\
					
					Оптимальное значение функции цели: C = 10.99999999999999
	\section{Обоснование оптимальности решений}
		Обратимся к теореме: допустимое базисное решение системы определяет крайнюю точку области, заданной неравенствами.\\
		Эта теорема позволяет утверждать следующее: если задача имеет оптимальное решение, то существует хотя бы одно оптимальное базисное решение.\\ Симплекс-метод может быть определен теперь как метод, предполагающий последовательный анализ базисных решений системы.\\
		Предварительно исследуем вопрос существования решений, симплекс метод исключает необходимость таких исследований, т. е. он не связан с априорными предположениями о свойствах систем. Симплекс- метод он состоит из двух этапов. Для 1-го этапа характерно следующее: начальные действия производятся с системой, входящей составной частью в стандартную форму задачи, причем не нужно ничего знать об этой системе или предварительно ее преобразовывать; в результате либо окажется найденным исходное базисное решение системы, либо последует вывод об отсутствии решения рассматриваемой задачи линейного программирования.\\
		Если возможность решения подтверждена, начинается второй этап реализации симплекс-метода, приводящий к отысканию экстремума.\\
		Обратимся теперь к теореме: допустимое базисное решение $x_i$ = $B_i$ (i = 1,...,m), $x_j$ = 0 (j = m + 1,....,n) является оптимальным тогда, когда коэффициенты $c_j$ при не базисных переменных в z = $z_0$ + $\sum_{j = m+ 1}^{n} c_jx_j$ не отрицательны.\\
		Из теоремы следует: всякое решение системы, которому соответствует нулевая сумма (z - $z_0$ = 0) при $\forall$ $c_j$ > 0 и $\forall$$x_j$ >= 0 (j = m + 1,..., n), также будет оптимальным. Наконец, допустимое оптимальное базисное решение окажется единственным, если $\forall c_j > 0$.
	\section{Приложения}
		\href{https://github.com/LuciusGen/MetOpt/tree/master/venv}{Код лаборатрной}
	
\end{document}